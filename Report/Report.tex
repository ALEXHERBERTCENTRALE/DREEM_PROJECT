\documentclass{article}

%\begin{figure}[h]
%\begin{center}
%\caption{\label{1} titre}
%\includegraphics[scale=0.5]{{nom_photo}.png}
%\captionsetup{labelformat=empty}
%\caption{ sources : nom_source}
%\end{center}
%\end{figure}

% \begin{center}
% \begin{tabular}{|c|c|}
			% \hline
			% a & a \\
			
			% a & a  \\
			% \hline
			% a & a \\
			
			% a & a  \\
			% \hline
			% a & a \\
			
			% a & a  \\
			% \hline
% \end{tabular}
% \end{center}

%[ForwardSearch("%bm.pdf","%Wc",%l,0,0,1)]


\usepackage[utf8]{inputenc}
\usepackage[T1]{fontenc}
\usepackage[french]{babel}
\usepackage{amsmath}
\usepackage{amssymb}
\usepackage{graphicx}
\usepackage{caption}
\usepackage{subcaption}
\usepackage{fancyhdr}
\usepackage{gensymb}
\usepackage{array}
\usepackage{caption}
%\usepackage[left=2cm,right=2cm,top=2cm,bottom=2cm]{geometry}
%\usepackage[left=2cm,right=2cm,top=2cm,bottom=2cm]{geometry}
%\newgeometry{top=1in,bottom=1in,right=0.5in,left=1.5in}

\usepackage{listings}
\usepackage{color}

\definecolor{dkgreen}{rgb}{0,0.6,0}
\definecolor{gray}{rgb}{0.5,0.5,0.5}
\definecolor{mauve}{rgb}{0.58,0,0.82}

\lstset{frame=tb,
  language=Python,
  aboveskip=3mm,
  belowskip=3mm,
  showstringspaces=false,
  columns=flexible,
  basicstyle={\small\ttfamily},
  numbers=none,
  numberstyle=\tiny\color{gray},
  keywordstyle=\color{blue},
  commentstyle=\color{dkgreen},
  stringstyle=\color{mauve},
  breaklines=true,
  breakatwhitespace=true,
  tabsize=3
}

\renewcommand{\thesection}{\Roman{section}}
\renewcommand{\thesubsection}{\arabic{subsection}}


\newcommand{\deriv}[2]{\frac{\partial #1}{\partial #2}}
\DeclareMathOperator{\Tr}{Tr}
\title{Machine learning pour la classification de phases de sommeil}
\author{Alexandre Herbert et Baptiste Turpin}
\date{}
\begin{document}

\maketitle

\newpage

\setcounter{tocdepth}{2}
\renewcommand{\contentsname}{Sommaire}
\tableofcontents


\section{Contexte}

\section{Choix d'implémentation}
\subsection{Choix du modèle}
Random Forest tout ça tout ça
\subsection{Sélection des méthodes de validation algorithmique}
cross validation toussa toussa


\section{Pré-traitement des données}
on a normalisé, ce genre de trucs
Après selection des features ?

\section{Selection de features}
Parler des observations faites sur les plots (featuring jolies images) 
Et puis aussi on prend les fft. Pourquoi ? Pourquoi pas.
\begin{itemize}
\item obs 1 
\item obs 2
\end{itemize}

D'où features potentiellement pertinentes (discriminantes) :
 \begin{center}
 \begin{tabular}{|c|c|}
			 \hline
			 a & a \\
			
			 a & a  \\
			 \hline
			 a & a \\
			
			 a & a  \\
			 \hline
			 a & a \\
			
			 a & a  \\
			 \hline
 \end{tabular}
 \end{center}

\section{Optimisation des hyperparamètres des features}
coucou, on a croisé les hyperparamètres et des choses stytlées.
Comme on est critiques et forts, on remarque qu'on risque d'overfitter sur les data de test en faisant ça !

\section{Résultats}

\section{ Critiques et perspectives}

\begin{lstlisting}
X_test_fft = h5py.File('data/X_test_fft.h5')

def buildAndSaveMatrix(h5file_freq, methodOne, param, list_bool_extract_signal, name_save):
    rep = extractFeatureAll(h5file_freq , methodOne , param , list_bool_extract_signal)
    temp_var_file = open("design_matrix/elem/" + name_save + '.txt','wb')
    pickle.dump(rep , temp_var_file)
    temp_var_file.close()
		len(bidule)
\end{lstlisting}


\end{document}
